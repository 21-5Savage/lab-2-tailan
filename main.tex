%   Доорх хэсгийг өөрчлөх шаардлагагүй
%----------------------------------------------------------------------------------------
\documentclass{report}
\usepackage{graphicx}
\usepackage{caption}
\usepackage{subcaption}
%!TEX TS-program = lualatex
%!TEX encoding = UTF-8 Unicode
\usepackage{placeins}
\usepackage{fontspec}
\setmainfont[Ligatures=TeX]{Times New Roman}
\setsansfont{Arial}

% \usepackage[utf8x]{inputenc}
% \usepackage[mongolian]{babel}
%\usepackage{natbib}
\usepackage{geometry}
%\usepackage{fancyheadings} fancyheadings is obsolete: replaced by fancyhdr. JL
\usepackage{fancyhdr}
\usepackage{float}
\usepackage{afterpage}
\usepackage{graphicx}
\usepackage{amsmath,amssymb,amsbsy}
\usepackage{dcolumn,array}
\usepackage{tocloft}
\usepackage{dics}
\usepackage{nomencl}
\usepackage{upgreek}
\newcommand{\argmin}{\arg\!\min}
\usepackage{mathtools}
\usepackage[hidelinks]{hyperref}
% Prevent hyperref warnings when it encounters \uppercase in titles etc.
% Make \uppercase a harmless no-op when hyperref builds PDF string/bookmark text.
\pdfstringdefDisableCommands{%
    % Make \uppercase expand to its argument when creating PDF strings
    \def\uppercase#1{#1}%
}

\usepackage{algorithm}
\usepackage{algpseudocode}

\usepackage{listings}
\DeclarePairedDelimiter\abs{\lvert}{\rvert}%
\makeatletter
\usepackage{caption}
\captionsetup[table]{belowskip=0.5pt}
\usepackage{subfiles}

\usepackage{listings}
\renewcommand{\lstlistingname}{Код}
\renewcommand{\lstlistlistingname}{\lstlistingname ын жагсаалт}

\usepackage{color}
\definecolor{codegreen}{rgb}{0,0.6,0}
\definecolor{codegray}{rgb}{0.5,0.5,0.5}
\definecolor{codepurple}{rgb}{0.58,0,0.82}
\definecolor{backcolour}{rgb}{0.99,0.99,0.99}
 
\lstdefinestyle{mystyle}{
    basicstyle=\ttfamily\small,
    backgroundcolor=\color{backcolour},   
    commentstyle=\color{codegreen},
    keywordstyle=\color{magenta},
    numberstyle=\tiny\color{codegray},
    stringstyle=\color{codepurple},
    %basicstyle=\footnotesize,
    breakatwhitespace=false,         
    breaklines=true,                 
    captionpos=b,                    
    keepspaces=false,                 
    numbers=left,                    
    numbersep=10pt,                  
    showspaces=false,                
    showstringspaces=true,
    showtabs=false,                  
    tabsize=2
}
 
% Use Consolas for monospace (better Cyrillic coverage) and avoid global listing label
\setmonofont{Consolas}
\lstset{style=mystyle}

\let\oldabs\abs
\def\abs{\@ifstar{\oldabs}{\oldabs*}}
\makenomenclature
% Ensure header has enough space for fancyhdr
\setlength{\headheight}{13pt}

% (No-op fallbacks removed — the package `dics.sty` defines the user-level
% commands such as \titleEng, \supervisor and \cosupervisor; we call
% \cosupervisor{} below if the document doesn't set a co-supervisor so that
% the internal macro \@cosupervisor is created.)

% Recognize common graphics extensions (including .png used in project)
\DeclareGraphicsExtensions{.pdf,.png,.jpg,.jpeg,.png,.pngf}
\begin{document}


%----------------------------------------------------------------------------------------
%   Өөрийн мэдээллээ оруулах хэсэг
%----------------------------------------------------------------------------------------

% Дипломийн ажлын сэдэв
\title{Хэв танилтын үндэс (ICSI380)}
% Дипломын ажлын англи нэр
\titleEng{Лаборатори №2}
% Өөрийн овог нэрийг бүтнээр нь бичнэ
\author{Бадарчийн Бат-Энх}
% Өөрийн овгийн эхний үсэг нэрээ бичнэ
\authorShort{Б. Бат-Энх}
% Удирдагчийн зэрэг цол овгийн эхний үсэг нэр
\supervisor{Др. Б. Сувдаа}
% Хамтарсан удирдагчийн зэрэг цол овгийн эхний үсэг нэр

% СиСи дугаар 
\sisiId{22B1NUM7226}
% Их сургуулийн нэр
\university{МОНГОЛ УЛСЫН ИХ СУРГУУЛЬ}
% Бүрэлдэхүүн сургуулийн нэр
\faculty{МЭДЭЭЛЛИЙН ТЕХНОЛОГИ ЭЛЕКТРОНИКИЙН СУРГУУЛЬ}
% Тэнхимийн нэр
\department{МЭДЭЭЛЭЛ, КОМПЬЮТЕРИЙН УХААНЫ ТЭНХИМ}
% Зэргийн нэр
\degreeName{Лабораторийн даалгавар}
% Суралцаж буй хөтөлбөрийн нэр
\programeName{Компьютерын Ухаан (D061301)}
% Хэвлэгдсэн газар
\cityName{Улаанбаатар}
% Хэвлэгдсэн огноо
\gradyear{2025 оны 10 сар}


%----------------------------------------------------------------------------------------
%   Доорх хэсгийг өөрчлөх шаардлагагүй
%----------------------------------------------------------------------------------------
% If no cosupervisor was provided, call the package macro with empty arg so
% that \@cosupervisor is defined and the title page code won't error.
\cosupervisor{}
\include{main-pre}

% Удиртгалыг оруулж ирэх ба abstract.tex файлд удиртгалаа бичнэ
\begin{abstract}
\hspace{1cm}Компьютерийн ухааны нэг хэсэг болох хэв танилт нь судалгааны ажил болон өдөр тутмын амьдралд хэрэглэгдэх чухал сэдвүүдийг судалсаар ирсэн билээ. Хэв танилт сэдэв нь өнөөдрийн эрчимтэй хөгжиж буй талбарууд болох Машин сургалт, Хиймэл оюуны судалгааны суурь онол болдог учир өнөөдрийг хүртэл хэв танилтын судалгааны чиглэл эрчимтэй явагдсаар байна. Харийн дүрс боловсруулалт нь хэв танилтын судалгааны томоохон суурь сэдвүүдийн нэг бөгөөд хэв танилтын хамгийн хуучин талбаруудын нэг гэдгээрээ онцлогтой. 

\section*{Зорилго}
\hspace{1cm}Энэхүү лабораторийн хүрээнд "Хэв танилтын үндэс" хичээлийн хүрээнд судалсан дүрс боловсруулах арга техникүүдийг практикт хэрэгжүүлж дүн шинжилгээ хийнэ. Түүнчлэн дүрс боловсруулалтын үндсэн суурь ойлголтуудыг бэхжүүлж, бодит асуудалд тохируулан параметр турших, үр дүнг үнэлэх чадварыг хөгжүүлэхэд тусална. 
    
    \hspace{1cm}Тус лабораторийг гүйцэтгэхдээ python хэл болон python хэл дээрх OpenCV, NumPy сангуудыг ашигласан болно
\section*{Зорилт}
\hspace{1cm}Дээрх зорилгийн хүрээнд дүрс боловсруулалтын алгоритм болон шинжлэх аргачлалуудыг тус бүрт нь дараах байдлаар судалж үзнэ.
\begin{itemize}
\item Онолын судалгаа: Ажиллах зарчмыг ойлгож бататган хэрэгжүүлэх алхмыг хөнгөвчлөх.
\item Хэрэгжүүлэлт: Онолын ойлголтыг практикт ашиглаж болох аргуудыг судалж хэрэгжүүлнэ.
\item Дүгнэлт: Хэрэгжүүлэлтийн үр дүнг нээлттэй сангуудын хэрэгжүүлэлт болон оновчтой үр дүнтэй харьцуулан үр дүнг шинжилнэ.
\end{itemize}
\end{abstract}


%----------------------------------------------------------------------------------------
%   Дипломын үндсэн хэсэг эндээс эхэлнэ
%----------------------------------------------------------------------------------------
%\addcontentsline{toc}{part}{БҮЛГҮҮД}
% Шинэ бүлэг
\chapter{Онолын судалгаа}
\subfile{onol.tex}

\chapter{Хэрэгжүүлэлт}
\subfile{prac.tex}

\chapter{Үр дүн}
\subfile{pic.tex}

\chapter{Код ба алгоритм оруулах}
Код оруулахдаа begin\{lstlisting\}  ... end\{lstlisting\} командын хооронд бичнэ.


%----------------------------------------------------------------------------------------
%   Дүгнэлт эндээс эхэлнэ
%----------------------------------------------------------------------------------------
\conclusion{Дүгнэлт}
Дүгнэлтийг энд бич

%----------------------------------------------------------------------------------------
%   Дипломын номзүй, хавсралтын хэсэг эндээс эхэлнэ
%----------------------------------------------------------------------------------------

\singlespace
\addcontentsline{toc}{part}{НОМ ЗҮЙ}
\begin{thebibliography}{}
	% Ашигласан материалыг эндээс оруулна
	\bibitem{image1}
	Inserting Images, Share LaTex, \url{https://www.sharelatex.com/learn/Inserting_Images}
	\bibitem{pharagraph1}
	Paragraphs and new lines,  Share LaTex, \url{https://www.sharelatex.com/learn/Paragraphs_and_new_lines}
	\bibitem{format1}
	Bold, italics and underlining, Share LaTex, \url{https://www.sharelatex.com/learn/Bold,_italics_and_underlining}
	\bibitem{list}
	Lists, Share LaTex, \url{https://www.sharelatex.com/learn/Lists}
    \bibitem{table}
    Tables, Share LaTex, https://www.sharelatex.com/learn/Tables
\end{thebibliography}


%----------------------------------------------------------------------------------------
%   Хавсралтууд эндээс эхэлнэ
%----------------------------------------------------------------------------------------
\appendix
\addcontentsline{toc}{part}{ХАВСРАЛТ}

% Хавсралтын нэр. Хавсралт гэдэг үг агуулахгүй

% Хавсралтын нэр. Хавсралт гэдэг үг агуулахгүй


\end{document}
