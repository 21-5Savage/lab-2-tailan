\begin{abstract}
\hspace{1cm}Компьютерийн ухааны нэг хэсэг болох хэв танилт нь судалгааны ажил болон өдөр тутмын амьдралд хэрэглэгдэх чухал сэдвүүдийг судалсаар ирсэн билээ. Хэв танилт сэдэв нь өнөөдрийн эрчимтэй хөгжиж буй талбарууд болох Машин сургалт, Хиймэл оюуны судалгааны суурь онол болдог учир өнөөдрийг хүртэл хэв танилтын судалгааны чиглэл эрчимтэй явагдсаар байна. Харийн дүрс боловсруулалт нь хэв танилтын судалгааны томоохон суурь сэдвүүдийн нэг бөгөөд хэв танилтын хамгийн хуучин талбаруудын нэг гэдгээрээ онцлогтой. 

\section*{Зорилго}
\hspace{1cm}Энэхүү лабораторийн хүрээнд "Хэв танилтын үндэс" хичээлийн хүрээнд судалсан дүрс боловсруулах арга техникүүдийг практикт хэрэгжүүлж дүн шинжилгээ хийнэ. Түүнчлэн дүрс боловсруулалтын үндсэн суурь ойлголтуудыг бэхжүүлж, бодит асуудалд тохируулан параметр турших, үр дүнг үнэлэх чадварыг хөгжүүлэхэд тусална. 
    
    \hspace{1cm}Тус лабораторийг гүйцэтгэхдээ python хэл болон python хэл дээрх OpenCV, NumPy сангуудыг ашигласан болно
\section*{Зорилт}
\hspace{1cm}Дээрх зорилгийн хүрээнд дүрс боловсруулалтын алгоритм болон шинжлэх аргачлалуудыг тус бүрт нь дараах байдлаар судалж үзнэ.
\begin{itemize}
\item Онолын судалгаа: Ажиллах зарчмыг ойлгож бататган хэрэгжүүлэх алхмыг хөнгөвчлөх.
\item Хэрэгжүүлэлт: Онолын ойлголтыг практикт ашиглаж болох аргуудыг судалж хэрэгжүүлнэ.
\item Дүгнэлт: Хэрэгжүүлэлтийн үр дүнг нээлттэй сангуудын хэрэгжүүлэлт болон оновчтой үр дүнтэй харьцуулан үр дүнг шинжилнэ.
\end{itemize}
\end{abstract}
