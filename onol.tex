\section{Spatial and Intensity Resolution}
Spatial resolution нь тухайн зургийн хамгийн жижиг нэгжийн дурсэлж чадах хэсгийн хэлдэг. Өндөр spatial resolution-тэй зураг нь жижиг элементүүдийг нарийвчлалтайгаар харуулах боломжтой. Intensity resolution нь пиксел бүр хэдэн өөр өнгө, туяаг илэрхийлэхийг заадаг бөгөөд өндөр intensity resolution-тай зураг гэрлийн ялгарлыг илүү нарийн үзүүлнэ. Эдгээр хоёр үзүүлэлт нь зураг боловсруулах, дүрсийг сайжруулах, анализ хийх үндсэн шалгуур болдог. Эдгээр хэмжүүрүүд нь анагаахын зураг болон астрономийн зайнаас авах зурагт зайлшгүй хяналт сайжруулалтад байх ёстой үзүүлэлтүүд юм.

\section{Image Interpolation}
Image interpolation нь зургийг масштаблаж хэмжээс өөрчлөгдөх үед тухайн зургийн гол "feature"-үүдийг хадгалж үлдэх зорилготой хэрэглэгддэг. Interpolation хамгийн энгийн бөгөөд түгээмэл арга нь nearest-neighbor арга юм. Хамгийн ойрын пикселийн утгыг шинээр үүсэх пикселд хуваарилах бөгөөд хурдан боловч жигд бус хэт хүрц ирмэгтэй зураг үүсгэж болно. Bilinear interpolation нь ойрын пикселүүдээс жингийн дундажийг авч жигд үр дүн гаргахыг зоридог. Bicubic interpolation нь арван зургаан пикселээс жингийн дундажийг тооцоолж, ирмэгийг хадгалах ба нарийвчилсан мэдээллийг илүү сайн хадгална. Энэхүү арга нь дүрс сайжруулах процесст өргөн хэрэглэгддэг.

\section{Contrast Stretching}
Contrast stretching нь зургийг бүрдүүлж буй гэрэлтэй ба бараан хэсгүүдээс тодорхой мужид орших утгуудыж илүү тодотгон харуулах арга юм. Бага contrast-той буюу бүрсгэх зургуудын элемэнтүүлийг илрүүлэхэд чухал хэрэгтэй техник юм. Мөн зураг дах ялгаралтыг нэмэгдүүлснээр, segmentation болон edge detection зэрэг ахисан түвшний боловсруулалтуудын бэлтгэл алхам болдог.

\section{Gray-Level Slicing}
Gray-level slicing нь зурагт тодорхой intensity дэх мужийг бусад мужаас салган дүрслэх техник юм. Энэ арга нь тодорхой муж дахь пикселүүдийг цайруулж эсвэл тод өнгөтэй болгож, бусад пикселүүдийг хэвээр үлдээх эсвэл бараан болгох зарчмаар ажиллана. Мөн тодруулах явцад background-ийг хэр өөрчилж буйгаас хамаарч хадгалах буюу арилгах гэсэн хоёр төрөлд хуваан авч үздэг. Gray level slicing техник нь зурагт тодорхой объект, үл мэдэгдэх хэсгийг илрүүлэхэд үр дүнтэй тул рентген зураг, хэт авиан дүрс сайжруулахад ашиглагддаг.

\section{Bit-plane Slicing}
Bit-plane slicing нь пикселийн бит бүрийг мужид хуваан салгах техник юм. Чимээ үүсгэж буй пикселүүд ихэвчлэн нэг өнгөний мужид оршдог тул feature агуулж буй пиксел болон чимээ үүсгэж буй пикселүүдийг салгахад үр дүнтэй байдаг. Иймд bit plane slicing техник нь дүрсийн мэдээлэлд нууц мэдээ нуух болон илрүүлэхэд ашиглагддаг. 

\section{Histogram Processing}
Histogram нь зургийн пикселүүдийн тархалтыг графикаар илэрхийлэх арга юм. Энэ нь тухайн дүрсэд дүн шинжилгээ хийхэд ашиглагддаг бөгөөд дотроо өөр өөр хэрэглээтэй олон аргачлалд салдаг. Жишээ нь, Энгийн histogram, пикселүүдийн утгын авч тархалтыг харуулах бол Normalized histogram нь аливаа пикселийн авч болох утгын магадлалыг харуулдаг. Түүнчлэн histogram дээр үйлдэл хийснээр эх зурагт боловсруулалт хийх боломжтой болдог. Жишээ нь, "outlier" чимээ илрүүлж хасах, хэт тод болон хэт бүдэг пикселүүдийн утгыг өөрчлөх. 

\section{Local Enhancement}
Local enhancement нь тухайн пикселийн ойрын хүрээтэй харьцуулан тэдгээрийн дундаж болон стандарт хазайлтад үндэслэн боловсруулалт хийх техник юм.  Жишээ нь 3×3 хүрээ ашиглан төвийн пикселийг орчны утгатай харьцуулж эх дүрсэд шингэсэн feature-үүдийг салгаж авдаг. Local enhancement нь дүрсийн пикселүүд нийтдээ өргөн хүрээг хамарсан үед ашиглагддаг.

\section{Enhancement using Arithmetic / Logic Operations}
Зургийг арифметик буюу логик үйлдлээр сайжруулах  гэдэг нь пикселийн тоон утга дээрх үйлдэл юм. Logic үйлдлүүд болон AND, OR, XOR, NOT-ийг ашиглан маск үүсгэж зургыг хэсэгчлэн салгах боломжтой. Энэ аргыг ашиглан зурагт background нэмэх, хасах мөн хоёр зураг нийлүүлэх, зургын ялгааг олох гэх мэт үйлдлүүдыг хийж болдог.

\section{Smoothing Linear Filters}
Smoothing linear filters нь зургийн чимээг багасгах, дүрс дэх ирмэгүүдийг зөөлрүүлэх арга юм. Averaging kernel эсвэл Gaussian kernel ашиглан бүдэгрүүлэхдээ төвийн пикселийг ойрын пикселүүдийн дундажтай сольдог. Kernel-ийн хэмжээ ихсэх тусам бүдгэрэлт ихсэж, чимээ багасдаг боловч ирмэгүүд дагаад бүдгэрэх учир feature extraction хүндрэлтэй болох магадлалтай. Smooth linear filter-ийг noise reduction, edge detection-ийн бэлтгэл шат болгон ашигладаг. 

\section{Median Filter}
Median filter smooth linear filter-тэй төстэй зарчимтай бөгөөд дундаж утга ашиглахын оронд медиан утгыг ашигладаг. Ингэснээр хэт тод эсвэл бүдэг пикселүүдийн нөлөөлөл багасдаг. Иймд salt-pepper төрлийн noise-ийг маш сайн арилгадаг. Энэ арга нь ирмэгийг хадгалах чадвартай тул linear averaging-тэй харьцуулахад илүү нарийвчилсан үр дүн өгдөг.

\section{Using Second-Derivative for Image Sharpening}
Зургийг хурцлахад ашиглагдах second-derivative буюу Laplacian операторууд нь дүрс дэх ирмэгүүдийг илрүүлж, гэрлийн хурц өөрчлөлтийг тодруулдаг. Laplacian нь хоёр дугаар зэргийн уламжлал авч, утга өөрчлөгдөх буюу уламжлалын тэмдэг өөрчлөгдөх мөчийг ирмэг гэж үзэж болно гэх онол дээр үндэслэн ажилладаг. Илэрсэн ирмэгийн лүрсийг эх дүрсэд нэмж хурцалсан зураг үүсгэдэг. Энэ арга нь spatial domain sharpening-д хамгийн түгээмэл бөгөөд edge detection болон texture enhancement-д ашиглагддаг.

\section{Unsharp Masking and High-boost Filtering}
Unsharp masking нь Gaussian blur-аас үүсэх дүрс болон эх дурс хоёрыг нэмж буюу давхарлана, харин High-boost filtering нь mask-ын коэффициентийг өндөрсгөж илүү хүчтэй хурцруулах боломж олгодог. Энэхүү арга нь local contrast ба ирмэгийг илрүүрэх процессыг хялбарчилж, бүрсгэр зургийг тодруулдаг. 

\section{Combining Spatial Enhancement Methods}
Олон spatial enhancement аргуудыг хослуулснаар дүрсний чанарыг хамгийн оновчтой байдлаар сайжруулж болно. Жишээ нь, Laplacian edge sharpening, histogram equalization, local enhancement-г нэгтгэж, зургийн ирмэг, гэрэл, contrast-ийг нэг дор сайжруулах эсвэл эх дүрсээс эхлэн сайжруулсан дүрс хүртлэх автомат Pipeline үүсгэх гэх мэтээр хэрэгжүүлж болно.  Composite methods нь ашиглан уян хатан байдлаар боловсруулалт хийх нь ихэнх тохиолдолд хамгийн ашигтай үр дүхг гаргана. 

\section{Notch Filters}
Notch filter нь давтамжийн орчинд тодорхой давтамжийн чимээг арилгах зориулалттай. Fourier transform ашиглан давтамжийн дүрслэлийг гаргаж, чимээ болон periodic noise оршин буй цэгүүдийг олж хасаад Inverse Fourier Transform-г ашиглан пикселийн орчинд эргэн хөрвүүлж noise багассан зургийг гаргадаг. 
Notch filter нь давтамжийн орчинд тодорхой давтамжийн чимээг арилгах хэрэглэгддэг.